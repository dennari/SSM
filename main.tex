\documentclass[12pt,a4paper,oneside,article]{memoir}



% LAYOUT
\usepackage{changepage}
\setulmarginsandblock{0.7\uppermargin}{0.8\lowermargin}{*}

% MATHS
\usepackage{amsmath,amsfonts,amssymb,amsbsy,commath,mathtools,calc}
\mathtoolsset{showonlyrefs,showmanualtags}

\usepackage{subfig}
% FONTS & LANGUAGE
\usepackage[usenames,dvipsnames]{color}
\definecolor{light-gray}{gray}{0.8}
\usepackage{fontspec,xltxtra,polyglossia}
\setmainlanguage{english}
\usepackage[normalem]{ulem} % have underlinings work
%\defaultfontfeatures{Ligatures=TeX}
\defaultfontfeatures{Mapping=tex-text}

\setmainfont[Ligatures={Common}, Numbers={OldStyle}]{Linux Libertine O}
%\setmainfont{Droid Sans}
%\setsansfont[Scale=MatchLowercase]{Inconsolata}
\setmonofont[Scale=0.8]{DejaVu Sans Mono}

% PDF SETUP
\usepackage[unicode,bookmarks, colorlinks, breaklinks,
pdftitle={T-61.3040: Ex 9},
pdfauthor={Ville Väänänen},
pdfproducer={xetex}
]{hyperref}
\hypersetup{linkcolor=black,citecolor=black,filecolor=black,urlcolor=MidnightBlue} 

% TABLES
\usepackage{booktabs}
\usepackage{topcapt} 
\usepackage{rccol}
\usepackage{tabularx} % requires array
\newcommand{\otoprule}{\midrule[\heavyrulewidth]}
\newcolumntype{d}[2]{R[.][.]{#1}{#2}}

\usepackage{titlesec}

%%%%%%%% OMAT KOMENNOT %%%%%%%%%%%%

\usepackage{mymath}
\usepackage{mylayout}


% PARAGRAPHS
%\usepackage{parskip}

% kuvat

\usepackage{listings}
\usepackage{mcode}
\lstset{ %
	%language=Matlab,                % choose the language of the code
	basicstyle=\footnotesize\ttfamily,% the size of the fonts that are used for the code 
	numbers=none,                   % where to put the line-numbers
	numberstyle=\footnotesize\ttfamily,      % the size of the fonts that are usedfor the line-numbers 
	stepnumber=5,                   % the step between two line-numbers. If it's 1 each line 
	aboveskip=2\medskipamount,
	belowskip=2\medskipamount,                                % will be numbered
	numbersep=-5pt,                  % how far the line-numbers are from the code
	backgroundcolor=\color{white},  % choose the background color. You must add \usepackage{color}
	showspaces=false,               % show spaces adding particular underscores
	showstringspaces=false,         % underline spaces within strings
	showtabs=false,                 % show tabs within strings adding particular underscores
	frame=l,
	framesep=0pt,
	framexleftmargin=2mm,
	rulecolor=\color{light-gray},	                % adds a frame around the code
	tabsize=2,	                % sets default tabsize to 2 spaces
	caption=,
	captionpos=t,                   % sets the caption-position to bottom
	breaklines=true,                % sets automatic line breaking
	breakatwhitespace=false,        % sets if automatic breaks should only happen at whitespace
	emptylines=*1,
	%title=\lstname,                 % show the filename of files included with
	                                % also try caption instead of title
	escapeinside={\%*}{*)},         % if you want to add a comment within your code
            % if you want to add more keywords to the set
}
 
\newcommand{\course}{T-61.3040}
\newcommand{\coursename}{Statistical Signal Modeling}
\newcommand{\duedate}{\today}
\newcommand{\studentid}{63527M}
\renewcommand{\title}{Exercise Round 9}
\author{Ville Väänänen}

\setsecnumdepth{subsubsection}
\counterwithout{section}{chapter}
\pagestyle{headings}
\makeevenhead{headings}{\course}{\Large\title}{\author / \studentid}
\makeoddhead{headings}{\course}{\Large\title}{\author / \studentid}
\makeheadrule{headings}{\textwidth}{\normalrulethickness}
\makeheadposition{headings}{flushleft}{flushleft}{flushleft}{flushleft}
\checkandfixthelayout
\renewcommand{\thesubsubsection}{\thesubsection{\large\scshape\alph{subsubsection}}}
\newfontfamily\subsubsectionfont[Letters=SmallCaps]{Linux Libertine O}
\titleformat{\subsection}{\large\scshape}{\alph{subsection} )}{10pt}{}
%\titleformat{\section}{\Huge}{Round \thesection}{10pt}{}
\titleformat{\section}{\Large}{Exercise \thesection}{10pt}{}

\everymath{\displaystyle}
\begin{document}
	\section*{Exercise 5}
\subsection{}
The periodogram of signal $x(n)$ with $N$ observations can be written
as
\begin{align}
	\hat{P}_x(e^{i\omega})&=\frac{1}{N}\abs{\sum_{n=0}^{N-1}x(n)e^{-in\omega}}^2
\end{align}
and for $\omega=0$ and $x(n)\in\R$ this gives
\begin{align}
	\hat{P}_x(1)&=\frac{1}{N}\left(\sum_{n=0}^{N-1}x(n)\right)^2
\end{align}


\subsection{}

\begin{align}
	\E{\hat{P}_x(1)}&=\frac{1}{N}\E{\left(\sum_{n=0}^{N-1}x(n)\right)^2}
\end{align}
and because $\E{x(n)^2}=1$ and $x(n)$ is white (the cross correlations are
zero), we get
\begin{align}
	\E{\hat{P}_x(1)}&=\frac{1}{N}\sum_{n=0}^{N-1}\E{x(n)^2}=1
\end{align}
For the variance we get
\begin{align}
	\var{\hat{P}_x(1)}&=\E{x(n)^2}-\E{x(n)}^2\\
	&=\frac{1}{N^2}\E{\left(\sum_{n=0}^{N-1}x(n)\right)^4}-1
\end{align}
Using the moment factoring theorem we can write
\begin{align}
	\E{\left(\sum_{n=0}^{N-1}x(n)\right)^4}&=\sum_{k=0}^{N-1}\sum_{l=0}^{N-1}\sum_{m=0}^{N-1}\sum_{n=0}^{N-1}\E{x(k)x(l)}\E{x(m)x(n)}\\
	&+\sum_{k=0}^{N-1}\sum_{l=0}^{N-1}\sum_{m=0}^{N-1}\sum_{n=0}^{N-1}\E{x(k)x(m)}\E{x(l)x(n)}\\
	&+\sum_{k=0}^{N-1}\sum_{l=0}^{N-1}\sum_{m=0}^{N-1}\sum_{n=0}^{N-1}\E{x(k)x(n)}\E{x(l)x(m)}
\end{align}
Inspecting these sums reveals that each term (or row in the previous expression)
gives $N^2$, so that we have
\begin{align}
	\var{\hat{P}_x(1)}&=\frac{3N^2}{N^2}-1=2
\end{align}


\subsection{}

In the Bartlett's method we divide the observed sequence into $N$ subsequences
and then estimate the power spectrum by taking the average
of the periodograms of the subsequences. Now our subsequence length is $L=1$,
giving the estimate
\begin{align}
	\hat{P}_x(e^{i\omega})&=\frac{1}{N}\sum_{n=0}^{N-1}x(n)^2
\end{align}
that is independent of the phase $\omega$.
The expecation is now
\begin{align}
	\E{\hat{P}_x(e^{i\omega})}&=\frac{1}{N}\sum_{n=0}^{N-1}\E{x(n)^2}=1
\end{align}
and the variance
\begin{align}
	\var{\hat{P}_x(e^{i\omega})}&=\frac{1}{N^2}\E{\left(\sum_{n=0}^{N-1}x(n)^2\right)^2}-1\\
	&=\frac{1}{N^2}\left(\sum_{m=0}^{N-1}\sum_{n=0}^{N-1}\E{x(m)^2}\E{x(n)^2}+2\E{x(m)x(n)}^2\right)-1\\
	&=\frac{1}{N^2}\left(N^2+2N\right)-1=\frac{2}{N}
\end{align}









\end{document}
