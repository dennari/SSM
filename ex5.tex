
\section*{Exercise 5}

Although it was not specified, we assume that $v(n)$ is white.

\subsection{}
The variance of $x(n)$ follows easily, as the variance of the sum
of uncorrelated variables is the sum of their variances and $\var{ax}=a^2\var{x}$
if $a$ is constant. We were also told that $\var{v(n)}=1$. We then get
\begin{align}
	\var{x(n)}&=b(0)\var{v(n)}+b(1)\var{v(n-1)}+b(2)\var{v(n-1)}\\
	&=b(0)^2+b(1)^2+b(2)^2\\
	&=1.53
\end{align}


\subsection{}

The result of filtering an input $v(n)$ with a LSI filter
can be computed from the convolution of the filter's impulse
response and the input:

\begin{align}
	x(n)&=h(n)*v(n)	
\end{align}
The impulse response of the MA-process is finite and
can be easily read from the coefficients. So we get:

\begin{align}
	x(n)&=\sum_{m=-\infty}^\infty \left(b(0)\delta(n-m)+b(1)\delta(n-1-m)+b(2)\delta(n-2-m)\right)v(m)	
\end{align}

Now if we think of filtering $x(n)$ again with $h(n)$, we could just
take the convolution again. However as the convolution operation
corresponds to multiplication of z-transforms, we get:

\begin{align}
	Y(z)=H(z)H(z)V(Z)=\left(b(0)+b(1)z^{-1}+b(2)z^{-2}\right)^2V(z)	
\end{align}
From here we can see that the result is a MA(4) process.

\subsection{}

We can simply calculate the difference equation and then
take the variance as in A):

\begin{align}
	y(n)&=\sum_{l=0}^2\sum_{m=0}^2 b(l)b(m)v(n-m-l)\\
	\Rightarrow \var{y(n)} &= \sum_{l=0}^2\sum_{m=0}^2 b(l)^2b(m)^2=\sum_{l=0}^2b(l)^2 \sum_{m=0}^2 b(m)^2\\
	&=\var{x(n)}^2=2.3409	
\end{align}
