\section*{Exercise 5}
\subsection{}
The periodogram of signal $x(n)$ with $N$ observations can be written
as
\begin{align}
	\hat{P}_x(e^{i\omega})&=\frac{1}{N}\abs{\sum_{n=0}^{N-1}x(n)e^{-in\omega}}^2
\end{align}
and for $\omega=0$ and $x(n)\in\R$ this gives
\begin{align}
	\hat{P}_x(1)&=\frac{1}{N}\left(\sum_{n=0}^{N-1}x(n)\right)^2
\end{align}


\subsection{}

\begin{align}
	\E{\hat{P}_x(1)}&=\frac{1}{N}\E{\left(\sum_{n=0}^{N-1}x(n)\right)^2}
\end{align}
and because $\E{x(n)^2}=1$ and $x(n)$ is white (the cross correlations are
zero), we get
\begin{align}
	\E{\hat{P}_x(1)}&=\frac{1}{N}\sum_{n=0}^{N-1}\E{x(n)^2}=1
\end{align}
For the variance we get
\begin{align}
	\var{\hat{P}_x(1)}&=\E{x(n)^2}-\E{x(n)}^2\\
	&=\frac{1}{N^2}\E{\left(\sum_{n=0}^{N-1}x(n)\right)^4}-1
\end{align}
Using the moment factoring theorem we can write
\begin{align}
	\E{\left(\sum_{n=0}^{N-1}x(n)\right)^4}&=\sum_{k=0}^{N-1}\sum_{l=0}^{N-1}\sum_{m=0}^{N-1}\sum_{n=0}^{N-1}\E{x(k)x(l)}\E{x(m)x(n)}\\
	&+\sum_{k=0}^{N-1}\sum_{l=0}^{N-1}\sum_{m=0}^{N-1}\sum_{n=0}^{N-1}\E{x(k)x(m)}\E{x(l)x(n)}\\
	&+\sum_{k=0}^{N-1}\sum_{l=0}^{N-1}\sum_{m=0}^{N-1}\sum_{n=0}^{N-1}\E{x(k)x(n)}\E{x(l)x(m)}
\end{align}
Inspecting these sums reveals that each term (or row in the previous expression)
gives $N^2$, so that we have
\begin{align}
	\var{\hat{P}_x(1)}&=\frac{3N^2}{N^2}-1=2
\end{align}


\subsection{}

In the Bartlett's method we divide the observed sequence into $N$ subsequences
and then estimate the power spectrum by taking the average
of the periodograms of the subsequences. Now our subsequence length is $L=1$,
giving the estimate
\begin{align}
	\hat{P}_x(e^{i\omega})&=\frac{1}{N}\sum_{n=0}^{N-1}x(n)^2
\end{align}
that is independent of the phase $\omega$.
The expecation is now
\begin{align}
	\E{\hat{P}_x(e^{i\omega})}&=\frac{1}{N}\sum_{n=0}^{N-1}\E{x(n)^2}=1
\end{align}
and the variance
\begin{align}
	\var{\hat{P}_x(e^{i\omega})}&=\frac{1}{N^2}\E{\left(\sum_{n=0}^{N-1}x(n)^2\right)^2}-1\\
	&=\frac{1}{N^2}\left(\sum_{m=0}^{N-1}\sum_{n=0}^{N-1}\E{x(m)^2}\E{x(n)^2}+2\E{x(m)x(n)}^2\right)-1\\
	&=\frac{1}{N^2}\left(N^2+2N\right)-1=\frac{2}{N}
\end{align}








