
\section*{Exercise 5}

\subsection{}
In the autocorrelation method we assume that the (real valued) signal is zero outside
the observed interval and the apply the Prony's method to determine the coefficients.
In this case where $p=1$, Prony's normal equation is
\begin{align}
		a(1)r_x(0)&=-r_x(1),\;k=1\\
		\Leftrightarrow a(1)&=-\frac{r_x(1)}{r_x(0)}
\end{align}
For the autocorrelation values we get:
\begin{align}
	r_x(k)&=\sum_{n=k}^1 x(n)x(n-k)	
\end{align}
giving
\begin{align}
	r_x(0)&=x(0)^2+x(1)^2\\
	r_x(1)&=x(1)x(0)	
\end{align}
Then
\begin{align}
	a(1)&=-\frac{x(1)x(0)}{x(0)^2+x(1)^2}	
\end{align}
which is defined as long as $x(0)\neq 0 $ or $x(1)\neq 0 $ (assumed in the exercise).  
According to the book we should have $b(0)=\sqrt{\varepsilon_1}$
where 
\begin{align}
	\varepsilon_1&=r_x(0)+a(1)r_x(1)	
\end{align}
is the minimum error. Thus we get
\begin{align}
	b(0)&=\sqrt{\frac{x(0)^4+x(0)^2x(1)^2+x(1)^4}{x(0)^2+x(1)^2}}	
\end{align}
The filter is stable if $\abs{a(1)}<1$, i.e
\begin{align}
		x(1)x(0) &< x(0)^2+x(1)^2
\end{align}

\subsection{}

In the covariance method we don't have to make any assumptions
of the values of the signal outside the observed interval, but we
modify the error that we're minimizing to include only terms we can
calculate. Similarly to the previous case we get

 \begin{align}
		a(1)&=-\frac{r_x(1,0)}{r_x(1,1)}
\end{align}
For the autocorrelation values we get:
\begin{align}
	r_x(k,l)&=x(1-l)x(1-k)	
\end{align}
giving
\begin{align}
	r_x(1,0)&=x(0)x(1)\\
	r_x(1,1)&=x(0)^2	
\end{align}
and
 \begin{align}
		a(1)&=-\frac{x(1)}{x(0)}
\end{align}
which is defined if $x(0)\neq 0$.
In this case we set $b(0)=x(0)$, so that our model is equal to the observation at time $0$.
This filter is stable if $\abs{x(1)}<\abs{x(0)}$.



